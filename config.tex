%%%%%%%%%%%%%%%%%%%%%%%%%%%%%%%%%%%%%%%%%%%%%%%%%%%%%%%%%%%%%%%%%%%%%%
%Package These (dans ~/Styles pour non-standards)
%%%%%%%%%%%%%%%%%%%%%%%%%%%%%%%%%%%%%%%%%%%%%%%%%%%%%%%%%%%%%%%%%%%%%%



%\usepackage[pagebackref]{hyperref}


\usepackage[a4paper,twoside,textwidth=14cm]{geometry}
\usepackage{amsmath,amsfonts,amssymb}      % classique
%\usepackage{frbib}                         % bib en fran{\c c}ais : fralpha
\usepackage{picins}                        % fig ds paragraphe
\usepackage{bm}                            % pour lettre bold en math
\usepackage{latexsym}                      % Latex symbole (pour \Box)
\usepackage{stmaryrd}                      % pour llbracket ...
\usepackage{url}                           % pour URL
\usepackage[francais]{babel}               
\usepackage[french]{minitoc}               % mini table des mati{\`e}res
\usepackage[dvips]{graphics}               % pour les graphics
\usepackage[dvips]{epsfig}
\usepackage{fancyhdr}                      % en tete, pied de page
\usepackage{subfigure}                     % sous figures

\usepackage{floatflt}                      % figure ds le texte
\usepackage{algorithm}                     % env algo
\usepackage{algorithmic}                   % float  algo



%\usepackage[ref]{ut-backref}                   %  backref

\usepackage{wrapfig}


\usepackage{authorindex}                   % index des auteurs
%\let\cite=\aicite  
                        % pour le package pr{\'e}c{\'e}dent
\usepackage{natbib}

\usepackage{multicol}

\usepackage[grey,utopia]{quotchap}         % pour num{\'e}ros chapitres en gris
\usepackage[twoside,figuresleft]{rotating} % turn table (pas dvi que ps)
\usepackage{french}                        % langue : javanais
\usepackage[T1]{fontenc}                   % pour coupure d'accent
%\usepackage{makeindex}
\usepackage{index}

%%Pour KDVI
%\usepackage[active]{srcltx}



\usepackage{colortbl}
\usepackage{color}
%\usepackage{backref}
%\usepackage[pagebackref]{hyperref}
%\hypersetup{
%  linktocpage,%
%  %%------------- Color Links ------------------------------ 
%  colorlinks=true,% 
%  linkcolor=myred,%
%  citecolor=mydarkblue,% 
% urlcolor=myblue,%
%  menucolor=red,%
%%  %%------------- Doc Info --------------------------------- 
%  pdftitle={Notes on non-rigid registration},%
%  pdfauthor={D. Sarrut},%
%%  %%------------ Doc View ----------------------------------}
%  pdfhighlight=/P,%
%  bookmarksopen=false,%
%  pdfpagemode=None}

%\hyperbaseurl{http://eric.univ-lyon2.fr/\string~dsarrut/bib/}

\definecolor{myblue}{rgb}{0,0,0.7}
\definecolor{myred}{rgb}{0.7,0,0}
\definecolor{mygreen}{rgb}{0,0.7,0}
\definecolor{mydarkblue}{rgb}{0,0,0.3} 

%%%%%%%%%%%%%%%%%%%%%%%%%%%%%%%%%%%%%%%%%%%%%%%%%%%%%%%%%%%%%%%%%%%%%%
%\bibliographystyle{fralpha}                % biblio en francais
\bibliographystyle{dav}
\graphicspath{{./Fig/}}                    % chemin des figs
%listfiles                                 % liste des fichiers a la compil
%%%%%%%%%%%%%%%%%%%%%%%%%%%%%%%%%%%%%%%%%%%%%%%%%%%%%%%%%%%%%%%%%%%%%%
\pagestyle{fancy}                          % pour fancychap
\fancyhf{}                                 % on vide les pieds de pages
\fancyhead{}                               % on vide les en-tete
\fancyhead[RO]{\slshape \rightmark}        % droite, page paire : section
\fancyhead[LE]{\slshape \leftmark}         % gauche, page impaire : chapitre
\fancyfoot[C]{-~\thepage~-}                % centre bas, num{\'e}ro de page

\renewcommand{\chaptermark}[1]{%           % chapitre en minuscule
  \markboth{\chaptername~%
    \thechapter.\ #1}{}}
\renewcommand{\sectionmark}[1]{%           % section en minuscule
  \markright{\thesection.\ #1}{}}


\fancypagestyle{plain}{%                   % red{\'e}finition style `plain'
  \fancyhf{}\fancyhead{}                   % rien en haut ni en bas
  \cfoot{-~\thepage~-}                     % centre bas, num{\'e}ro de page
  \renewcommand{\headrulewidth}{0pt}       % pas de ligne en haut
  \renewcommand{\footrulewidth}{0pt}       % pas de ligne en bas
 }
%%%%%%%%%%%%%%%%%%% Pour Algorithmic

%%%%%%%%%%%%%%%%%%%%%%%% D{\'e}but des red{\'e}finitions %%%%%%%%%%%%%%%%%%%%

% pour le package algorithm an fran{\c c}ais
\renewcommand{\listalgorithmname}{Table des algorithmes}
\floatname{algorithm}{Algorithme}

% pour le package algorithmic en fran{\c c}ais plus 3 commandes: RETURN, INTERNNAME et
% EXTERNNAME qui permettent respectivement de faire un return et de donner
% un nom {\`a} l'Algorithme avec un certain style et enfin de lui donner un nom 
% dans le cas o{\`u} il est r{\'e}cursif (seul cas, me
% semble-t-il, o{\`u} il soit vraiment n{\'e}cessaire de lui donner un
% nom..)
% Lorsque l'on veut donner un nom {\`a} l'algorithme, on place la commande
% \EXTERNNAME juste avant de lui donner son nom et en d{\'e}but
% d'algorithme. La commande INTERNNAME permet de le mettre en majuscule..
\def\RETURN{\bf retourner}
\newcommand{\EXTERNNAME}{\item[]\hspace{-1.5em}}
\newcommand{\INTERNNAME}[1]{\textsc{#1}}
\renewcommand{\listalgorithmname}{Table des algorithmes}
\renewcommand{\algorithmiccomment}[1]{\hspace{1.5em}\{\textsf{#1}\}}
\renewcommand{\algorithmicrequire}{\textbf{Requiert}}
\renewcommand{\algorithmicensure}{\textbf{Post-condition:}}
\renewcommand{\algorithmicend}{\textbf{fin du}}
\renewcommand{\algorithmicif}{\textbf{si}}
\renewcommand{\algorithmicthen}{\textbf{alors}}
\renewcommand{\algorithmicelse}{\textbf{sinon}}
\renewcommand{\algorithmicelsif}{\algorithmicelse\ \algorithmicif}
\renewcommand{\algorithmicendif}{\algorithmicend\ \algorithmicif}
%\renewcommand{\algorithmicendif}{}
\renewcommand{\algorithmicfor}{\textbf{pour}}
\renewcommand{\algorithmicforall}{\textbf{pour tout}}
\renewcommand{\algorithmicdo}{\textbf{faire}}
\renewcommand{\algorithmicendfor}{\algorithmicend\ \algorithmicfor}
%\renewcommand{\algorithmicendfor}{}
\renewcommand{\algorithmicwhile}{\textbf{tant que}}
\renewcommand{\algorithmicendwhile}{\algorithmicend\ \algorithmicwhile}
%\renewcommand{\algorithmicendwhile}{}
\renewcommand{\algorithmicloop}{\textbf{loop}}
\renewcommand{\algorithmicendloop}{\algorithmicend\ \algorithmicloop}
%\renewcommand{\algorithmicendloop}{}
\renewcommand{\algorithmicrepeat}{\qtextbf{R{\'e}p\'ter}}
\renewcommand{\algorithmicuntil}{\textbf{jusqu'{\`a} ce que}}

%%%%%%%%%%%%%%%%%%%%%%%%% Fin des red{\'e}finitions %%%%%%%%%%%%%%%%%%%%%%

% Pour algorithmic
%\newcommand{\EXTERNNAME}{\item[]\hspace{-1.5em}}
%\newcommand{\INTERNNAME}[1]{\textsc{#1}}
%%%%%%%%%%%%%%%%%%%%%%%%%%%%%%%%%%%%%%%%%%%%%%%%%%%%%%%%%%%%%%%%%%%%%%
\newenvironment{mapreuve}%                 % environnement preuve
{\begin{description}%               
\item [Preuve :] \sl }{ \hfill $\Box$%     % termine par carr{\'e} blanc
\end{description} }

\newtheorem{theo}{Th{\'e}or{\`e}me}[chapter]           % environnement th{\'e}or{\`e}me
\newlength{\xvtextwidth}
\xvtextwidth\textwidth 
\advance\xvtextwidth - 4cm

%\renewcommand{\emph}[1]{#1\scalebox{0.4}{\includegraphics{dav.eps}}}

\newtheorem{defi}{D{\'e}finition}[chapter]
\newtheorem{lem}{Lemme}[chapter]
\newtheorem{coro}{Corollaire}[chapter]
\newtheorem{prop}{Proposition}[chapter]
\newtheorem{conj}{Conjecture}[chapter]
\newtheorem{example}{Exemple}[chapter]
%%%%%%%%%%%%%%%%%%%%%%%%%%%%%%%%%%%%%%%%%%%%%%%%%%%%%%%%%%%%%%%%%%%%%%
 % \nomtcrule
% \renewcommand{\mtctitle}{Sommaire\hrule}
\newcommand{\mychaptoc}[1]{%               % chapitre+minitoc
  \chapter{#1}                         
  \thispagestyle{plain}
  %\centerline{\Large $\Diamond$}

  \minitoc

  %\centerline{\Large $\Diamond$}
  %\flushright$\Box$
  \newpage}                                % saut de page
%%%%%%%%%%%%%%%%%%%%%%%%%%%%%%%%%%%%%%%%%%%%%%%%%%%%%%%%%%%%%%%%%%%%%%
\setcounter{secnumdepth}{3}      % depth of numbering of sectionning commands
\setcounter{tocdepth}{1}         % depth of table of contents
\raggedbottom                    % or \flushbottom, at your choice
\setcounter{lofdepth}{1}         % pour la liste de figures : subfig aussi
\setcounter{minitocdepth}{3}     % profondeur minitoc



%%%%%%%%%%%%%%%%%%%%%%%%%%%%%%%%%%%%%%%%%%%%%%%%%%%%%%%%%%%%%%%%%%%%%%
\newcommand{\round}[1]{\lceil #1 \rfloor}  % notation arrondi
\def\eme{^{\textrm{{\`e}me}}}                  % i {\`e}me
\def\num{n^{\circ}}                        % numero
\def\Num{N^{\circ}}                        % Numero
\def\sinc{\mathrm{sinc}}                   % sinus cardinal
\def\ere{$^{\textrm{{\`e}re}}$}                % {\`e}re
\def\eg{\emph{e.g.} }                      % e.g.
\def\ie{\emph{i.e.} }                      % i.e.
\def\etc{\emph{etc}}                       % etc
\def\cm{\,cm}                              % cm
\def\met{\,m}                              % m
\def\mm{\,mm}                              % mm
\def\deg{$^\circ$}                         % degres
\def\ud{\mathrm{d}}                        % pour dx dy ...

\def \R {{\Bbb R}}
\def \Z {{\mathbb Z}}
\def \L {{\mathcal{L}}}
\def \C {{\mathcal C}}
\def \P {{\mathcal P}}
\def \Q {{\mathcal Q}} 
\def \E{{\mathcal E}}
\def \D{{\mathcal D}}
\def \BD {{\bar{\mathcal{D}}}}
\def \etal {{\it et al.~}}
\def\arc{\mbox{arc}}



\newcommand{\aut}[1]{{\sc #1}}             % auteur en small capsu

\newcommand{\mat}[1]{\left( \begin{array}{cccc} #1 \end{array} \right)}
\newcommand{\matdd}[1]{\left( \begin{array}{cc} #1 \end{array} \right)}
\newcommand{\tabfdiv}[3]{\mbox{#1}  & #2 & =  #3} 

%%%%%%%%%%%%%%%%%%%%%%%%%%%%%%%%%%%%%%%%%%%%%%%%%%%%%%%%%%%%%%%%%%%%%%

\usepackage{pstchap}



\newcommand{\monchapter}[1]{%

  \chapter{#1}

 \vfill\minitoc\vfill\vfill

\clearpage}

%\newcommand{\moncitet}[1]{
%%   \citet{#1}
%   \index{\citeauthor{#1}}
%}
\renewcommand{\listtablename}{Table des tableaux}

%ICI
%\makeindex


%\newindex{cite}{ctx}{cnd}{Index des auteurs cit{\'e}s}
%\renewcommand{\citeindextype}{cite}


% C'est pour les index avec NatBib

%\citeindextrue


%%% Local Variables: 
%%% mode: latex
%%% TeX-master: "these.tex"
%%% End: 
