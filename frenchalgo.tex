%%%%%%%%%%%%%%%%%%%%%%%% D{\'e}but des red{\'e}finitions %%%%%%%%%%%%%%%%%%%%

% pour le package algorithm an fran{\c c}ais
\renewcommand{\listalgorithmname}{Table des algorithmes}
\floatname{algorithm}{Algorithme}

% pour le package algorithmic en fran{\c c}ais plus 3 commandes: RETURN, INTERNNAME et
% EXTERNNAME qui permettent respectivement de faire un return et de donner
% un nom {\`a} l'Algorithme avec un certain style et enfin de lui donner un nom 
% dans le cas o{\`u} il est r{\'e}cursif (seul cas, me
% semble-t-il, o{\`u} il soit vraiment n{\'e}cessaire de lui donner un
% nom..)
% Lorsque l'on veut donner un nom {\`a} l'algorithme, on place la commande
% \EXTERNNAME juste avant de lui donner son nom et en d{\'e}but
% d'algorithme. La commande INTERNNAME permet de le mettre en majuscule..
\def\RETURN{\bf retourner}
\newcommand{\EXTERNNAME}{\item[]\hspace{-1.5em}}
\newcommand{\INTERNNAME}[1]{\textsc{#1}}
\renewcommand{\listalgorithmname}{Table des algorithmes}
\renewcommand{\algorithmiccomment}[1]{\hspace{1.5em}\{\textsf{#1}\}}
\renewcommand{\algorithmicrequire}{\textbf{Requiert}}
\renewcommand{\algorithmicensure}{\textbf{Post-condition:}}
\renewcommand{\algorithmicend}{\textbf{fin du}}
\renewcommand{\algorithmicif}{\textbf{si}}
\renewcommand{\algorithmicthen}{\textbf{alors}}
\renewcommand{\algorithmicelse}{\textbf{sinon}}
\renewcommand{\algorithmicelsif}{\algorithmicelse\ \algorithmicif}
\renewcommand{\algorithmicendif}{\algorithmicend\ \algorithmicif}
%\renewcommand{\algorithmicendif}{}
\renewcommand{\algorithmicfor}{\textbf{pour}}
\renewcommand{\algorithmicforall}{\textbf{pour tout}}
\renewcommand{\algorithmicdo}{\textbf{faire}}
\renewcommand{\algorithmicendfor}{\algorithmicend\ \algorithmicfor}
%\renewcommand{\algorithmicendfor}{}
\renewcommand{\algorithmicwhile}{\textbf{tant que}}
\renewcommand{\algorithmicendwhile}{\algorithmicend\ \algorithmicwhile}
%\renewcommand{\algorithmicendwhile}{}
\renewcommand{\algorithmicloop}{\textbf{loop}}
\renewcommand{\algorithmicendloop}{\algorithmicend\ \algorithmicloop}
%\renewcommand{\algorithmicendloop}{}
\renewcommand{\algorithmicrepeat}{\qtextbf{R{\'e}p\'ter}}
\renewcommand{\algorithmicuntil}{\textbf{jusqu'{\`a} ce que}}

%%%%%%%%%%%%%%%%%%%%%%%%% Fin des red{\'e}finitions %%%%%%%%%%%%%%%%%%%%%%
