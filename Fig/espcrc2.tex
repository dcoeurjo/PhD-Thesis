%%%%%%%%%% espcrc2.tex %%%%%%%%%%
%
% $Id: espcrc2.tex 1.2 2000/07/24 09:12:51 spepping Exp spepping $
%
\documentclass[fleqn,twoside]{article}
\usepackage{espcrc2}

% change this to the following line for use with LaTeX2.09
% \documentstyle[twoside,fleqn,espcrc2]{article}

% if you want to include PostScript figures
\usepackage{graphicx}
% if you have landscape tables
\usepackage[figuresright]{rotating}

% put your own definitions here:
%   \newcommand{\cZ}{\cal{Z}}
%   \newtheorem{def}{Definition}[section]
%   ...
\newcommand{\ttbs}{\char'134}
\newcommand{\AmS}{{\protect\the\textfont2
  A\kern-.1667em\lower.5ex\hbox{M}\kern-.125emS}}

% add words to TeX's hyphenation exception list
\hyphenation{author another created financial paper re-commend-ed Post-Script}

% declarations for front matter
\title{Elsevier instructions for the preparation of a
       2-column format camera-ready paper in \LaTeX}

\author{P. de Groot\address[MCSD]{Mathematics and Computer Science Section, 
        Elsevier Science B.V., \\ 
        P.O. Box 103, 1000 AC Amsterdam, The Netherlands}%
        \thanks{Footnotes should appear on the first page only to
                indicate your present address (if different from your
                normal address), research grant, sponsoring agency, etc.
                These are obtained with the {\tt\ttbs thanks} command.},
        R. de Maas\addressmark\thanks{For following authors with the same
                address use the {\tt\ttbs addressmark} command.},
        X.-Y. Wang\address{Economics Department, University of Winchester, \\
        2 Finch Road, Winchester, Hampshire P3L T19, United Kingdom}
        and
        A. Sheffield\addressmark[MCSD]\thanks{To reuse an addressmark
                later on, label the address with an optional argument to the
                {\tt \ttbs address} command, e.g. {\tt\ttbs
                address[MCSD]}, and repeat the label
                as the optional argument to the {\tt\ttbs addressmark}
                command, e.g. {\tt\ttbs addressmark[MCSD]}.}}
       
\begin{document}

\begin{abstract}
These pages provide you with an example of the layout and style for
100\% reproduction which we wish you to adopt during the preparation of
your paper. This is the output from the \LaTeX{} document class you
requested.
\vspace{1pc}
\end{abstract}

% typeset front matter (including abstract)
\maketitle

\section{FORMAT}

Text should be produced within the dimensions shown on these pages:
each column 7.5 cm wide with 1 cm middle margin, total width of 16 cm
and a maximum length of 19.5 cm on first pages and 21 cm on second and
following pages. The \LaTeX{} document class uses the maximum stipulated
length apart from the following two exceptions (i) \LaTeX{} does not
begin a new section directly at the bottom of a page, but transfers the
heading to the top of the next page; (ii) \LaTeX{} never (well, hardly
ever) exceeds the length of the text area in order to complete a
section of text or a paragraph. Here are some references:
\cite{Scho70,Mazu84}.

\subsection{Spacing}

We normally recommend the use of 1.0 (single) line spacing. However,
when typing complicated mathematical text \LaTeX{} automatically
increases the space between text lines in order to prevent sub- and
superscript fonts overlapping one another and making your printed
matter illegible.

\subsection{Fonts}

These instructions have been produced using a 10 point Computer Modern
Roman. Other recommended fonts are 10 point Times Roman, New Century
Schoolbook, Bookman Light and Palatino.

\section{PRINTOUT}

The most suitable printer is a laser or an inkjet printer. A dot
matrix printer should only be used if it possesses an 18 or 24 pin
printhead (``letter-quality'').

The printout submitted should be an original; a photocopy is not
acceptable. Please make use of good quality plain white A4 (or US
Letter) paper size. {\em The dimensions shown here should be strictly
adhered to: do not make changes to these dimensions, which are
determined by the document class}. The document class leaves at least
3~cm at the top of the page before the head, which contains the page
number.

Printers sometimes produce text which contains light and dark streaks,
or has considerable lighting variation either between left-hand and
right-hand margins or between text heads and bottoms. To achieve
optimal reproduction quality, the contrast of text lettering must be
uniform, sharp and dark over the whole page and throughout the article.

If corrections are made to the text, print completely new replacement
pages. The contrast on these pages should be consistent with the rest
of the paper as should text dimensions and font sizes.

\section{TABLES AND ILLUSTRATIONS}

Tables should be made with \LaTeX; illustrations should be originals or
sharp prints. They should be arranged throughout the text and
preferably be included {\em on the same page as they are first
discussed}. They should have a self-contained caption and be positioned
in flush-left alignment with the text margin within the column. If they
do not fit into one column they may be placed across both columns
(using \verb-\begin{table*}- or \verb-\begin{figure*}- so that they
appear at the top of a page).

\subsection{Tables}

Tables should be presented in the form shown in
Table~\ref{table:1}.  Their layout should be consistent
throughout.

\begin{table*}[htb]
\caption{The next-to-leading order (NLO) results
{\em without} the pion field.}
\label{table:1}
\newcommand{\m}{\hphantom{$-$}}
\newcommand{\cc}[1]{\multicolumn{1}{c}{#1}}
\renewcommand{\tabcolsep}{2pc} % enlarge column spacing
\renewcommand{\arraystretch}{1.2} % enlarge line spacing
\begin{tabular}{@{}lllll}
\hline
$\Lambda$ (MeV)           & \cc{$140$} & \cc{$150$} & \cc{$175$} & \cc{$200$} \\
\hline
$r_d$ (fm)                & \m1.973 & \m1.972 & \m1.974 & \m1.978 \\
$Q_d$ ($\mbox{fm}^2$)     & \m0.259 & \m0.268 & \m0.287 & \m0.302 \\
$P_D$ (\%)                & \m2.32  & \m2.83  & \m4.34  & \m6.14  \\
$\mu_d$                   & \m0.867 & \m0.864 & \m0.855 & \m0.845 \\
$\mathcal{M}_{\mathrm{M1}}$ (fm)   & \m3.995 & \m3.989 & \m3.973 & \m3.955 \\
$\mathcal{M}_{\mathrm{GT}}$ (fm)   & \m4.887 & \m4.881 & \m4.864 & \m4.846 \\
$\delta_{\mathrm{1B}}^{\mathrm{VP}}$ (\%)   
                          & $-0.45$ & $-0.45$ & $-0.45$ & $-0.45$ \\
$\delta_{\mathrm{1B}}^{\mathrm{C2:C}}$ (\%) 
                          & \m0.03  & \m0.03  & \m0.03  & \m0.03  \\
$\delta_{\mathrm{1B}}^{\mathrm{C2:N}}$ (\%) 
                          & $-0.19$ & $-0.19$ & $-0.18$ & $-0.15$ \\
\hline
\end{tabular}\\[2pt]
The experimental values are given in ref. \cite{Eato75}.
\end{table*}

\begin{sidewaystable}
\caption{The next-to-leading order (NLO) results
{\em without} the pion field.}
\label{table:2}
\newcommand{\m}{\hphantom{$-$}}
\newcommand{\cc}[1]{\multicolumn{1}{c}{#1}}
\renewcommand{\arraystretch}{1.2} % enlarge line spacing
\begin{tabular*}{\textheight}{@{\extracolsep{\fill}}lllllllllllll}
\hline
& $\Lambda$ (MeV) & \cc{$140$} & \cc{$150$} & \cc{$175$} & \cc{$200$} & \cc{$225$} & \cc{$250$} &
\cc{Exp.} & \cc{$v_{18}$~\cite{v18}} &  \\
\hline
%b
 & $r_d$ (fm)                        & \m1.973 & \m1.972 & \m1.974 & \m1.978 & \m1.983 & \m1.987 & 1.966(7) & \m1.967 & \\[2pt]
 & $Q_d$ ($\mbox{fm}^2$)             & \m0.259 & \m0.268 & \m0.287 & \m0.302 & \m0.312 & \m0.319 & 0.286    & \m0.270 & \\[2pt]
 & $P_D$ (\%)                        & \m2.32  & \m2.83  & \m4.34  & \m6.14  & \m8.09  & \m9.90  & $-$      & \m5.76  & \\[2pt]
 & $\mu_d$                           & \m0.867 & \m0.864 & \m0.855 & \m0.845 & \m0.834 & \m0.823 & 0.8574   & \m0.847 & \\[5pt]
 & $\mathcal{M}_{\mathrm{M1}}$ (fm)             & \m3.995 & \m3.989 & \m3.973 & \m3.955 & \m3.936 & \m3.918 & $-$      & \m3.979 & \\[5pt]
 & $\mathcal{M}_{\mathrm{GT}}$ (fm)             & \m4.887 & \m4.881 & \m4.864 & \m4.846 & \m4.827 & \m4.810 & $-$      & \m4.859 & \\[2pt]
 & $\delta_{\mathrm{1B}}^{\mathrm{VP}}$ (\%)   & $-0.45$ & $-0.45$ & $-0.45$ & $-0.45$ & $-0.45$ & $-0.44$ & $-$      & $-0.45$ & \\[2pt]
 & $\delta_{\mathrm{1B}}^{\mathrm{C2:C}}$ (\%) & \m0.03  & \m0.03  & \m0.03  & \m0.03  & \m0.03  & \m0.03  & $-$      & \m0.03  & \\[2pt]
 & $\delta_{\mathrm{1B}}^{\mathrm{C2:N}}$ (\%) & $-0.19$ & $-0.19$ & $-0.18$ & $-0.15$ & $-0.12$ & $-0.10$ & $-$      & $-0.21$ & \\
\hline
\end{tabular*}\\[2pt]
The experimental values are given in ref. \cite{Eato75}.
\end{sidewaystable}

Horizontal lines should be placed above and below table headings, above
the subheadings and at the end of the table above any notes. Vertical
lines should be avoided.

If a table is too long to fit onto one page, the table number and
headings should be repeated above the continuation of the table. For
this you have to reset the table counter with
\verb|\addtocounter{table}{-1}|. Alternatively, the table can be turned
by $90^\circ$ (`landscape mode') and spread over two consecutive pages
(first an even-numbered, then an odd-numbered one) created by means of
\verb|\begin{table}[h]| without a caption. To do this, you prepare the
table as a separate \LaTeX{} document and attach the tables to the
empty pages with a few spots of suitable glue.

\subsection{Useful table packages}

Modern \LaTeX{} comes with several packages for tables that
provide additional functionality. Below we mention a few. See
the documentation of the individual packages for more details. The
packages can be found in \LaTeX's \texttt{tools} directory.

\begin{description}
  
\item[\texttt{array}] Various extensions to \LaTeX's \texttt{array}
  and \texttt{tabular} environments.
  
\item[\texttt{longtable}] Automatically break tables over several
  pages. Put the table in the \texttt{longtable} environment instead
  of the \texttt{table} environment.
  
\item [\texttt{dcolumn}] Define your own type of column. Among others,
  this is one way to obtain alignment on the decimal point.

\item[\texttt{tabularx}] Smart column width calculation within a
  specified table width.
  
\item[\texttt{rotating}] Print a page with a wide table or figure in
  landscape orientation using the \texttt{sidewaystable} or
  \texttt{sidewaysfigure} environments, and many other rotating
  tricks. Use the package with the \texttt{figuresright} option to
  make all tables and figures rotate in clockwise. Use the starred
  form of the \texttt{sideways} environments to obtain full-width
  tables or figures in a two-column article.

\end{description}

\subsection{Line drawings}

Line drawings may consist of laser-printed graphics or professionally
drawn figures attached to the manuscript page. All figures should be
clearly displayed by leaving at least one line of spacing above and
below them. When placing a figure at the top of a page, the top of the
figure should align with the bottom of the first text line of the other
column.

Do not use too light or too dark shading in your figures; too dark a
shading may become too dense while a very light shading made of tiny
points may fade away during reproduction.

All notations and lettering should be no less than 2\,mm high. The use
of heavy black, bold lettering should be avoided as this will look
unpleasantly dark when printed.

\subsection{PostScript figures}

Instead of providing separate drawings or prints of the figures you
may also use PostScript files which are included into your \LaTeX{}
file and printed together with the text. Use one of the packages from
\LaTeX's \texttt{graphics} directory: \texttt{graphics},
\texttt{graphicx} or \texttt{epsfig}, with the \verb|\usepackage|
command, and then use the appropriate commands
(\verb|\includegraphics| or \verb|\epsfig|) to include your PostScript
file.

The simplest command is: \newline
\verb|\includegraphics{file}|, which inserts the
PostScript file \texttt{file} at its own size. The starred version of
this command: \newline
\verb|\includegraphics*{file}|, does the same, but clips
the figure to its bounding box.

With the \texttt{graphicx} package one may specify a series of options
as a key--value list, e.g.:
\begin{tabular}{@{}l}
\verb|\includegraphics[width=15pc]{file}|\\
\verb|\includegraphics[height=5pc]{file}|\\
\verb|\includegraphics[scale=0.6]{file}|\\
\verb|\includegraphics[angle=90,width=20pc]{file}|
\end{tabular}

See the file \texttt{grfguide}, section ``Including Graphics Files'',
of the \texttt{graphics} distribution for all options and a detailed
description.

The \texttt{epsfig} package mimicks the commands familiar from the
package with the same name in \LaTeX2.09. A PostScript file
\texttt{file} is included with the command
\verb|\psfig{file=file}|.

Grey-scale and colour photographs cannot be included in this way,
since reproduction from the printed CRC article would give
insufficient typographical quality. See the following subsections.

\begin{figure}[htb]
\vspace{9pt}
\framebox[55mm]{\rule[-21mm]{0mm}{43mm}}
\caption{Good sharp prints should be used and not (distorted) photocopies.}
\label{fig:largenenough}
\end{figure}
%
\begin{figure}[htb]
\framebox[55mm]{\rule[-21mm]{0mm}{43mm}}
\caption{Remember to keep details clear and large enough.}
\label{fig:toosmall}
\end{figure}

\subsection{Black and white photographs}

Photographs must always be sharp originals ({\em not screened
versions\/}) and rich in contrast. They will undergo the same reduction
as the text and should be pasted on your page in the same way as line
drawings.

\subsection{Colour photographs}

Sharp originals ({\em not transparencies or slides\/}) should be
submitted close to the size expected in publication. Charges for the
processing and printing of colour will be passed on to the author(s) of
the paper. As costs involved are per page, care should be taken in the
selection of size and shape so that two or more illustrations may be
fitted together on one page. Please contact the Author Support
Department at Elsevier (E-mail: \texttt{authorsupport@elsevier.nl})
for a price quotation and layout instructions before producing your
paper in its final form.

\section{EQUATIONS}

Equations should be flush-left with the text margin; \LaTeX{} ensures
that the equation is preceded and followed by one line of white space.
\LaTeX{} provides the document class option {\tt fleqn} to get the
flush-left effect.

\begin{equation}
H_{\alpha\beta}(\omega) = E_\alpha^{(0)}(\omega) \delta_{\alpha\beta} +
                          \langle \alpha | W_\pi | \beta \rangle 
\end{equation}

You need not put in equation numbers, since this is taken care of
automatically. The equation numbers are always consecutive and are
printed in parentheses flush with the right-hand margin of the text and
level with the last line of the equation. For multi-line equations, use
the {\tt eqnarray} environment.

For complex mathematics, use the \AmS math package. This package
sets the math indentation to a positive value. To keep the equations
flush left, either load the \texttt{espcrc} package \emph{after} the
\AmS math package or set the command \verb|\mathindent=0pt| in the
preamble of your article.

\begin{thebibliography}{9}
\bibitem{Scho70} S. Scholes, Discuss. Faraday Soc. No. 50 (1970) 222.
\bibitem{Mazu84} O.V. Mazurin and E.A. Porai-Koshits (eds.),
                 Phase Separation in Glass, North-Holland, Amsterdam, 1984.
\bibitem{Dimi75} Y. Dimitriev and E. Kashchieva, 
                 J. Mater. Sci. 10 (1975) 1419.
\bibitem{Eato75} D.L. Eaton, Porous Glass Support Material,
                 US Patent No. 3 904 422 (1975).
\end{thebibliography}

References should be collected at the end of your paper. Do not begin
them on a new page unless this is absolutely necessary. They should be
prepared according to the sequential numeric system making sure that
all material mentioned is generally available to the reader. Use
\verb+\cite+ to refer to the entries in the bibliography so that your
accumulated list corresponds to the citations made in the text body. 

Above we have listed some references according to the
sequential numeric system \cite{Scho70,Mazu84,Dimi75,Eato75}.
\end{document}
