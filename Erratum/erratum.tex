\documentclass{article}
\usepackage{amsmath,amssymb}
\usepackage[T1]{fontenc}


\title{Erratum}
\author{David Coeurjolly}

\begin{document}
\maketitle

\begin{itemize}

\item {\bf page 14 : Fig1.8} : il y a une inversion des codages de
  Freeman entre la 4- et la 8- adjacence (par rapport � la figure 1.7)

\item {\bf page 41  : Algorithme 5,  lignes 28-32} : il faut supprimer
  le ``sinon''  dans la ligne  30.  En effet  si $b=1$, alors les deux
  tests sont v�rifi�s en m�me temps. Dans ce cas, il faut bien d�caler
  $U'$ ET $L'$.

\item {\bf page 45 : Tableau 2.2} : le co�t en m�moire des algorithmes
  de Preparata-Shamos, Vittone et Vittone modifi�  est en $O(n)$ et
  non en $O(n^2)$. 
\item {\bf page 51} : $g.(-3,17,5)=\ldots \Rightarrow
  (-3,17,5).g=\ldots$ (merci � Isabelle Sivignon)
\item {\bf page 62 : Definition de $\mathcal{P}$} :
  $\mathcal{P}=\{(x,y,z)\in\mathbb{Z}^3~|~0\leq \alpha x+\beta y+ \gamma - z
  <1\}$ ($-z$ au lieu de $+z$)
\item {\bf page 65 : Preuve prop. 2.8} : m�me erreur, correction: 
\begin{align*}
C_1&:\quad\alpha x+\beta y +\gamma -z\geq 0\\
C_2&:\quad\alpha x+\beta y +\gamma -z<1
\end{align*}

%\item {\bf page 73, Prop 2.9} : la probabilit� d'avoir le code ``0''
%  est $\frac{b}{a+b}$, celle d'avoir le code ``1'' est
%  $\frac{a}{a+b}$. (merci � Bertrand Kerautret et Damien Jamet).

\item {\bf page 77, Prop 2.10} : les  probabilit�s sont les suivantes:
  \begin{align*}
    Type1&: \frac{c}{a+b+c}\\
    Type2&: \frac{b}{a+b+c}\\
    Type3&: \frac{a}{a+b+c}
  \end{align*}

(merci  � Bertrand Kerautret et Damien Jamet).


\end{itemize}


\end{document}
