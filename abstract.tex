The context  of  the  work presented  in  this  thesis is the  digital
geometry.  This research area is devoted  to the automatic analysis of
objects in  digital images  in  dimension  2 and 3.    All acquisition
devices  provide data organized  on regular grids, called {\it digital
data}.   The  algorithms  that  are  explored  and  extended keep  the
discrete aspect of  the data, in opposition to  techniques based on an
approximation process of a continuous model.

More precisely, we  are interested in the  study of digital curves and
surfaces.  First of all,  we  consider basic  digital objects such  as
digital  straight lines, planes  and  circles.  We present  algorithms
that allow to characterize such objects and we propose some extensions
of these methods.  Then, we study some metrics  on the digital objects
such  as the Euclidean  distance transform  and  the notion of digital
geodesic.   An approach based   on the visibility property  in digital
domains is presented.    In the third  part,  we define and   evaluate
estimators of the Euclidean measurements such as the length, the curvature
or the area.  Some results on the  convergence of these estimators are
presented.  Finally, we illustrate some applications in which
these  researches have been  used for: archaeological object automatic
classification and snow sample micro-structure analysis.
